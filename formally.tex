% !TEX root = ./type-inference-paper.tex
\begin{figure}[t]
$\arraycolsep=4pt\begin{array}{lll}
\tau & ::= &
  \tnum ~\vert~
  \tarr{\tau}{\tau} ~\vert~
  \ehole^A
  \\
e & ::= &
  x ~\vert~
  \lamfunc{x}{e} ~\vert~
  \lamfunc{x:\tau}{e} ~\vert~
  e(e) ~\vert~
  \underlinenum{n} ~\vert~
  (e+e) ~\vert~
  e: \tau ~\vert~
  \ehole^u  ~\vert~
  \notehole{e}^u \\
C & :: = &
	C ~\vert~
	C \cup C \\
\end{array}$
\end{figure}

\begin{figure}[t]
  \fbox{$\consexptyp{\Gamma}{e}{\tau}{C}$}~~\text{$e$ synthesizes $\tau$}
\begin{subequations}
\begin{equation}
\inferrule[SVar]{ }{
  \consexptyp{\Gamma, x : \tau}{x}{\tau}{\econs}
}
\end{equation}
\begin{equation}
\inferrule[SNum]{ }{
  \consexptyp{\Gamma}{\hnum{n}}{\tnum}{\econs}
}
\end{equation}

\begin{equation}
\inferrule[SAp]{
  \consexptyp{\Gamma}{e_1}{\tau_1}{C_1} \\
  \tau_1 \typearrow \tarr{\tau_{in}}{\tau_{out}} \addcons{C_2} \\
  \ana{\Gamma}{e_2}{\tau_{in}}{C_3}
}{
  \consexptyp{\Gamma}{\hap{e_1}{e_2}}{\tau_{out}} { C_1 \cup C_2 \cup C_3}
}
\end{equation}
\begin{equation}
\inferrule[SPlus]{
  \ana{\Gamma}{e_1}{\tnum}{C_1} \\
  \ana{\Gamma}{e_2}{\tnum}{C_2}
}{
  \consexptyp{\Gamma}{(e_1 + e_2)}{\tnum}{C_1 \cup C_2}
}
\end{equation}

\begin{equation}
\inferrule[SAsc]{
  \ana{\Gamma}{e}{\tau}{C}
}{
  \consexptyp{\Gamma}{(e : \tau)}{\tau}{C}
}
\end{equation}

\begin{equation}
\inferrule[SEHole]{(A~\text{fresh}) }{
  \consexptyp{\Gamma}{\llparenthesiscolor \rrparenthesiscolor^u}{\llparenthesiscolor \rrparenthesiscolor^A}{\econs}
}
\end{equation}
\begin{equation}
\inferrule[SHole]{
 (A~\text{fresh}) \\
 \consexptyp{\Gamma}{e}{\tau}{C}
}{
  \consexptyp{\Gamma}{\llparenthesiscolor e \rrparenthesiscolor^u}{\llparenthesiscolor \rrparenthesiscolor^A}{C}
}
\end{equation}

\begin{equation}
\inferrule[SLamAnn]{
  \consexptyp{\Gamma, x : \tau_{in}}{e}{\tau_{out}}{C}
}{
  \consexptyp{\Gamma}{\lamfunc{x:\tau_{in}}{e}}{\tarr{\tau_{in}}{\tau_{out}}}{C}
}
\end{equation}
%\begin{equation}
%\inferrule[SLetAnn]{
% \ana{\Gamma}{e_1}{\tau_1}{C_1} \\
% \consexptyp{\Gamma,x : \tau_1}{e_2}{\tau_2}{C_2}
%}{
%  \consexptyp{\Gamma}{let \; x:\tau_1 \; be \; e_1 \; in \; e_2}{\tau_2}%{C_1 \cup C_2}
%}
%\end{equation}

% \begin{equation}
% \inferrule[SLet]{
%  \consexptyp{\Gamma}{e_1}{\tau_1}{C_1} \\
%  \consexptyp{\Gamma,x : \tau_1}{e_2}{\tau_2}{C_2}
% }{
%   \consexptyp{\Gamma}{let \; x \; be \; e_1 \; in \; e_2}{\tau_2}{C_1 \cup C_2}
% }
% \end{equation}

\end{subequations}
\end{figure}

\begin{figure}[t]
\fbox{$\ana{\Gamma}{e}{\tau} {C}$}~~\text{$e$ analyzes against $\tau$}
\begin{subequations}

\begin{equation}
\inferrule[ALamAnn]{
 \tau \typearrow \tarr{\tau_{in}}{\tau_{out}} \addcons{C_1} \\
  \ana{\Gamma, x : \tau'_{in}}{e}{\tau_{out}}{C_2} \\
  \tau_{in} \sim \tau'_{in}
}{
  \ana{\Gamma}{\lamfunc{x:\tau'_{in}}{e}}{\tau}{C_1 \cup C_2 \cup \{ \tau_{in} \sim \tau'_{in} \}}
}
\end{equation}

\begin{equation}
\inferrule[ALam]{
 \tau \typearrow \tarr{\tau_{in}}{\tau_{out}} \addcons{C_1} \\
  \ana{\Gamma, x : \tau_{in}}{e}{\tau_{out}}{C_2}
}{
  \ana{\Gamma}{\lamfunc{x}{e}}{\tau}{C_1 \cup C_2}
}
\end{equation}

\begin{equation}
\inferrule[ASubsumption]{
  \consexptyp{\Gamma}{e}{\tau'}{C_1} \\
  \tau \sim \tau' 
}{
  \ana{\Gamma}{e}{\tau}{C_1 \cup \{\tau \sim \tau'  \}}
}
\end{equation}

% \begin{equation}
% \inferrule[ALetAnn]{
%  \ana{\Gamma}{e_1}{\tau_1}{C_1} \\
%  \ana{\Gamma,x : \tau_1}{e_2}{\tau_2}{C_2}
% }{
%   \ana{\Gamma}{let \; x:\tau_1 \; be \; e_1 \; in \; e_2}{\tau_2}{C_1 \cup C_2}
% }
% \end{equation}

% \begin{equation}
% \inferrule[ALet]{
%  \consexptyp{\Gamma}{e_1}{\tau_1}{C_1} \\
%  \ana{\Gamma,x : \tau_1}{e_2}{\tau_2}{C_2}
% }{
%   \ana{\Gamma}{let \; x \; be \; e_1 \; in \; e_2}{\tau_2}{C_1 \cup C_2}
% }
% \end{equation}


\end{subequations}
\end{figure}


\begin{figure}[t]
\fbox{$\tau \typearrow \tarr{\tau_{in}}{\tau_{out}} \addcons{C}$}~~\text{$\tau$ has matching arrow type $\tarr{\tau_{in}}{\tau_{out}}$}
\begin{subequations}

\begin{equation}
\inferrule[ArrowMatchArrow]{ }{
  \tarr{\tau_{in}}{\tau_{out}} \typearrow \tarr{\tau_{in}}{\tau_{out}} \addcons{\econs}
}
\end{equation}
\newline
when $\tau$ is hole type:
\begin{equation}
\inferrule[ ]{
 (B~\text{fresh}) \\
 (C~\text{fresh})
}{
  \tehole^A \typearrow \tarr{\tehole^B}{\tehole^C} \addcons{\{ \tehole^A \sim \tarr{\tehole^B}{\tehole^C} \} }
}
\end{equation}

\end{subequations}
\end{figure}

\begin{figure}[t]
\fbox{$\tau \sim \tau $}~~\text{$\tau$ is consistent to $\tau$}
\begin{subequations}

\begin{equation}
\inferrule[CNum]{
}{
  num \sim num
}
\end{equation}


\begin{equation}
\inferrule[CDL]{
}{
  \tehole^A \sim \tau
}
\end{equation}

\begin{equation}
\inferrule[CDR]{
}{
  \tau \sim \tehole^A
}
\end{equation}

\begin{equation}
\inferrule[CArrow]{
\tau_1 \sim \tau_3 \\
\tau_2 \sim \tau_4
}{
  \tarr{\tau_1}{\tau_2} \sim \tarr{\tau_3}{\tau_4}
}
\end{equation}

\end{subequations}
\end{figure}

