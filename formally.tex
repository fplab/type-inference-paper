% !TEX root = ./type-inference-paper.tex
\begin{figure}[t]
$\arraycolsep=4pt\begin{array}{lll}
\tau & ::= &
  \tnum ~\vert~
  \tarr{\tau}{\tau} ~\vert~
  \ehole^A 
  \\
e & ::= &
  x ~\vert~
  \lamfunc{x}{e} ~\vert~
  e(e) ~\vert~
  \underlinenum{n} ~\vert~
  (e+e) ~\vert~
  e: \tau ~\vert~
  \ehole^u  ~\vert~
  \notehole{e}^u \\
\end{array}$
\end{figure}

\begin{figure}[t]
  \fbox{$\consexptyp{\Gamma}{e}{\tau}{C}$}~~\text{$e$ synthesizes $\tau$}
\begin{subequations}
\begin{equation}
\inferrule[SVar]{ }{
  \consexptyp{\Gamma, x : \tau}{x}{\tau}{\econs}
}
\end{equation}
\begin{equation}
\inferrule[SNum]{ }{
  \consexptyp{\Gamma}{\hnum{n}}{\tnum}{\econs}
}
\end{equation}

\begin{equation}
\inferrule[SAp]{
  \consexptyp{\Gamma}{e_1}{\tau_1}{C_1} \\
  \tau_1 \typearrow \tarr{\tau_{in}}{\tau_{out}} \addcons{C_2} \\
  \ana{\Gamma}{e_2}{\tau_{in}}{C_3}
}{
  \consexptyp{\Gamma}{\hap{e_1}{e_2}}{\tau_{out}} { C_1 \cup C_2 \cup C_3}
}
\end{equation}
\begin{equation}
\inferrule[SPlus]{
  \ana{\Gamma}{e}{\tnum}{C_1} \\
  \ana{\Gamma}{e}{\tnum}{C_2}
}{
  \consexptyp{\Gamma}{(e_1 + e_2)}{\tnum}{C_1 \cup C_2}
}
\end{equation}

\begin{equation}
\inferrule[*]{
  \ana{\Gamma}{e}{\tau}{C}
}{
  \consexptyp{\Gamma}{(e : \tau)}{\tau}{C}
}
\end{equation}

\begin{equation}
\inferrule[SEHole]{(A~fresh) }{
  \consexptyp{\Gamma}{\llparenthesiscolor \rrparenthesiscolor^u}{\llparenthesiscolor \rrparenthesiscolor^A}{\econs}
}
\end{equation}
\begin{equation}
\inferrule[SHole]{
 (A~fresh) \\
 \consexptyp{\Gamma}{e}{\tau}{C}
}{
  \consexptyp{\Gamma}{\llparenthesiscolor e \rrparenthesiscolor^u}{\llparenthesiscolor \rrparenthesiscolor^A}{C}
}
\end{equation}
\end{subequations}
\end{figure}

\begin{figure}[t]
\fbox{$\ana{\Gamma}{e}{\tau} {C}$}~~\text{$e$ analyzes against $\tau$}
\begin{subequations}

\begin{equation}
\inferrule[ALam]{
 \tau \typearrow \tarr{\tau_{in}}{\tau_{out}} \addcons{C_1} \\
  \ana{\Gamma, x : \tau_{in}}{e}{\tau_{in}}{C_2}
}{
  \ana{\Gamma}{\lamfunc{x}{e}}{\tau}{C_1 \cup C_2}
}
\end{equation}

\begin{equation}
\inferrule[ASubsumption]{
  \consexptyp{\Gamma}{e}{\tau'}{C_1} \\
  \tau \sim \tau'
}{
  \ana{\Gamma}{e}{\tau}{C_1 \cup \{\tau \simeq \tau' \}}
}
\end{equation}


\end{subequations}
\end{figure}


\begin{figure}[t]
\fbox{$\tau \typearrow \tarr{\tau_{in}}{\tau_{out}} \addcons{C}$}~~\text{$\tau$ is equivalent to $\tarr{\tau_{in}}{\tau_{out}}$}
\begin{subequations}

\begin{equation}
\inferrule[EType]{ }{
  \tarr{\tau_{in}}{\tau_{out}} \typearrow \tarr{\tau_{in}}{\tau_{out}} \addcons{\econs}
}
\end{equation}

\begin{equation}
\inferrule[EHole]{
 (B~fresh) \\
 (C~fresh)
}{
  \tehole^A \typearrow \tarr{\tehole^B}{\tehole^C} \addcons{\{ A \simeq B \rightarrow C \} }
}
\end{equation}

\end{subequations}
\end{figure}

\begin{thm}[Preservation]
  \label{thrm:preservation}
  If $\tau \sim \tau' \addcons{C}$ then $\tau \sim \tau'$
\end{thm}

\begin{thm}[Constraint Existence]
  \label{Constraint Existence}
  If $\Gamma \vdash e \Rightarrow \tau$ then $\consexptyp{\Gamma}{e}{\tau}{C}$ for some C
\end{thm}