\section{Type Hole Inference}
\label{sec:typinf}
\subsection{Syntax of Typed Lambda Calculus with Holes}
\label{sec:syntax}

Figure ~\ref{fig:syntax_fig} defines the syntax of H-types, H-expressions, and type constraint set. We start from the usual definition of H-types and H-expressions system [ref Live Cyrus], except two modifications. First, we add annotated lambda and extend synthesis rules in Figure [n]. Second, we add a unique \emph{type hole identifier} which is drawn as a superscript $A$ for each type hole $\ehole^A$. \emph{Type hole identifiers} are only used to distinguish different type holes in the unification (Sec. ~\ref{sec:infalg}). All hole types with different identifiers are type-consistent with each other. Type consistency $\tau \sim \tau$ is a symmetric and reflective but not transitive relation defined in Figure ~ref{fig:type-consistency} to show type equality between H-types [ref Hazelnut]. \par
The number expression drawn by $\underlinenum{n}$ corresponding to mathematical number $n$ is of $\tnum$ type, and we only define plus operation $e + e$ for simplicity. There are two kinds of hole expression: empty holes, $\ehole$, stands for missing part of an incomplete program, and non-empty holes, $\notehole{e}$, operating as membranes around static and dynamic type inconsistencies [ref Live Cyrus]. \par
\begin{figure}[htpb]
$\arraycolsep=4pt\begin{array}{lll}
\multicolumn{3}{l}{\hrulefill} \\
HTyp~~ \tau & ::= &
  \tnum ~\vert~
  \tarr{\tau}{\tau} ~\vert~
  \ehole^A
  \\
HExp~~ e & ::= &
  x ~\vert~
  \lamfunc{x}{e} ~\vert~
  \lamfunc{x:\tau}{e} ~\vert~
  e(e) ~\vert~
  \underlinenum{n} ~\vert~
  (e+e) ~\vert~
  e: \tau ~\vert~
  \ehole  ~\vert~
  \notehole{e} \\
C & :: = &
	C ~\vert~
	C \cup C \\
\multicolumn{3}{l}{\hrulefill} 
\end{array}$
\caption{Syntax of H-types, H-expressions, and Type Constraint Set}
\label{fig:syntax_fig}
\end{figure}

\emph{Type constraint set} $C$ is a set of type consistency equations called \emph{type constraints}. We add a union operation, $C \cup C$, corresponding to mathematical set union operation to generate constraints inductively through bidirectional propagation in Figure ~\ref{fig:ana-syn}. The type constraint set is solved by unification-based algorithm in Section ~\ref{sec:infalg} to infer types for type holes.

\subsection{Bidirectional Typing}
\label{sec:bidirectional}

\begin{figure}
    \begin{multicols}{2}
      \fbox{$\consexptyp{\Gamma}{e}{\tau}{C}$}~~\text{$e$ synthesizes $\tau$}\hfill
    \begin{subequations}
    \begin{equation}\label{rule:syn-var}
        \inferrule[]{ }{
            \consexptyp{\Gamma, x : \tau}{x}{\tau}{\econs}
          }
    \end{equation}
    \begin{equation}\label{rule:syn-num}
        \inferrule[]{ }{
            \consexptyp{\Gamma}{\hnum{n}}{\tnum}{\econs}
          }
    \end{equation}
    \begin{equation}\label{rule:syn-plus}
        \inferrule[]{
            \ana{\Gamma}{e_1}{\tnum}{C_1} \\
            \ana{\Gamma}{e_2}{\tnum}{C_2}
          }{
            \consexptyp{\Gamma}{(e_1 + e_2)}{\tnum}{C_1 \cup C_2}
          }
    \end{equation}
    \begin{equation}\label{rule:syn-asc}
        \inferrule[]{
            \ana{\Gamma}{e}{\tau}{C}
          }{
            \consexptyp{\Gamma}{(e : \tau)}{\tau}{C}
          }
    \end{equation}
    \begin{equation}\label{rule:syn-ehole}
        \inferrule[]{(A~\text{fresh}) }{
            \consexptyp{\Gamma}{\llparenthesiscolor \rrparenthesiscolor}{\llparenthesiscolor \rrparenthesiscolor^A}{\econs}
          }
    \end{equation}
    \begin{equation}\label{rule:syn-hole}
        \inferrule[]{
            (A~\text{fresh}) \\
            \consexptyp{\Gamma}{e}{\tau}{C}
           }{
             \consexptyp{\Gamma}{\llparenthesiscolor e \rrparenthesiscolor}{\llparenthesiscolor \rrparenthesiscolor^A}{C}
           }
    \end{equation}
    \begin{equation}\label{rule:syn-lamann}
        \inferrule[]{
          \consexptyp{\Gamma, x : \tau_{in}}{e}{\tau_{out}}{C}
        }{
          \consexptyp{\Gamma}{\lamfunc{x:\tau_{in}}{e}}{\tarr{\tau_{in}}{\tau_{out}}}{C}
        }
    \end{equation}
    \begin{equation}\label{rule:syn-ap}
      \inferrule[]{
          \consexptyp{\Gamma}{e_1}{\tau_1}{C_1} \\
          \tau_1 \typearrow \tarr{\tau_{in}}{\tau_{out}} \addcons{C_2} \\
          \ana{\Gamma}{e_2}{\tau_{in}}{C_3}
        }{
          \consexptyp{\Gamma}{\hap{e_1}{e_2}}{\tau_{out}} { C_1 \cup C_2 \cup C_3}
        }
  \end{equation}
    \end{subequations}
    \vspace{4px}\fbox{$\ana{\Gamma}{e}{\tau} {C}$}~~\text{$e$ analyzes against $\tau$}\hfill
    \begin{subequations}
    \begin{equation}\label{rule:ana-subsume}
      \inferrule[]{
          \consexptyp{\Gamma}{e}{\tau'}{C_1} \\
          \tau \sim \tau' 
        }{
          \ana{\Gamma}{e}{\tau}{C_1 \cup \{\tau \sim \tau'  \}}
        }
  \end{equation}
    \begin{equation}\label{rule:ana-lam}
        \inferrule[]{
            \tau \typearrow \tarr{\tau_{in}}{\tau_{out}} \addcons{C_1} \\
             \ana{\Gamma, x : \tau_{in}}{e}{\tau_{out}}{C_2}
           }{
             \ana{\Gamma}{\lamfunc{x}{e}}{\tau}{C_1 \cup C_2}
           }
    \end{equation}
    \begin{equation}\label{rule:ana-lamann}
        \inferrule[]{
         \tau \typearrow \tarr{\tau_{in}}{\tau_{out}} \addcons{C_1} \\
          \ana{\Gamma, x : \tau'_{in}}{e}{\tau_{out}}{C_2} \\
          \tau_{in} \sim \tau'_{in}
        }{
          \ana{\Gamma}{\lamfunc{x:\tau'_{in}}{e}}{\tau}{C_1 \cup C_2 \cup \{ \tau_{in} \sim \tau'_{in} \}}
        }
    \end{equation}
    \end{subequations}
  \end{multicols}
  \hrule
  \caption{H-type synthesis and analysis.}
  \label{fig:ana-syn}
\end{figure}

\begin{figure}
   \fbox{$\tau \sim \tau $}~~\text{$\tau$ is consistent to $\tau$}\hfill
    \begin{subequations}\label{eqns:consistency}
    \begin{mathpar}
      \hfill
        \inferrule[]{
            }{
              num \sim num
            }
            \hfill
    \inferrule[]{
        }{
        \tehole^A \sim \tau
        }
        \hfill
    \inferrule[]{
        }{
        \tau \sim \tehole^A
        }
        \hfill
    \inferrule[]{
        \tau_1 \sim \tau_3 \\
        \tau_2 \sim \tau_4
        }{
        \tarr{\tau_1}{\tau_2} \sim \tarr{\tau_3}{\tau_4}
        }\hfill \text{\hspace{-2px}(\ref*{eqns:consistency}a-d)}
    \end{mathpar}
  \end{subequations}
  \hrule
  \caption{H-type consistency.}
  \label{fig:type-consistency}
\end{figure}

\begin{figure}
    \fbox{$\tau \typearrow \tarr{\tau_{in}}{\tau_{out}} \addcons{C}$}~~\text{$\tau$ has matching arrow type $\tarr{\tau_{in}}{\tau_{out}}$}\hfill
    \begin{subequations}\label{eqns:matcharrow}
      \begin{minipage}{0.43\linewidth}
        \begin{equation}
          \inferrule[]{ }{
            \tarr{\tau_{in}}{\tau_{out}} \typearrow \tarr{\tau_{in}}{\tau_{out}} \addcons{\econs}
          }
        \end{equation}
        \end{minipage}
        \begin{minipage}{0.55\linewidth}
        \begin{equation}
          \inferrule[ ]{
            (B~\text{fresh}) \\
            (C~\text{fresh})
           }{
             \tehole^A \typearrow \tarr{\tehole^B}{\tehole^C} \addcons{\{ \tehole^A \sim \tarr{\tehole^B}{\tehole^C} \} }
           }
        \end{equation}
        \end{minipage}

    \end{subequations}
    \hrule
    \caption{Matched arrow types.}
    \label{fig:type-consistency}
  \end{figure}
