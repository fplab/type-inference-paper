\PassOptionsToPackage{svgnames,dvipsnames,svgnames}{xcolor}
\newif\ifarxiv
\arxivfalse
\ifarxiv
\documentclass[acmsmall,nonacm]{acmart}
\else
%%% Note: arxiv does not want line numbers (they are detected somehow, and are not allowed)
\documentclass[acmsmall,review,anonymous,nonacm]{acmart}
\settopmatter{printfolios=false,printccs=false,printacmref=false}
% \settopmatter{printccs=false,printacmref=false}
\fi
  
%% For single-blind review submission
%\documentclass[acmlarge,review]{acmart}\settopmatter{printfolios=true}
%% For final camera-ready submission
%\documentclass[acmlarge]{acmart}\settopmatter{}

%% Note: Authors migrating a paper from PACMPL format to traditional
%% SIGPLAN proceedings format should change 'acmlarge' to
%% 'sigplan,10pt'.

% \bibliographystyle{ACM-Reference-Format}


%% Some recommended packages.
\usepackage{booktabs}   %% For formal tables:
                        %% http://ctan.org/pkg/booktabs
\usepackage{subcaption} %% For complex figures with subfigures/subcaptions
                        %% http://ctan.org/pkg/subcaption

%% Cyrus packages
\usepackage{multirow}
\usepackage{multicol}
\usepackage{microtype}
\usepackage{mdframed}
\usepackage{colortab}
\usepackage{mathpartir}
\usepackage{enumitem}
\usepackage{bbm}
\usepackage{stmaryrd}
\usepackage{mathtools}
\usepackage{leftidx}
\usepackage{todonotes}
\usepackage{xspace}
\usepackage{wrapfig}
\usepackage{extarrows}
% \usepackage[subtle]{savetrees}
\usepackage{listings}%
\lstloadlanguages{ML}
\lstset{tabsize=2, 
basicstyle=\footnotesize\ttfamily, 
% keywordstyle=\sffamily,
commentstyle=\itshape\ttfamily\color{gray}, 
stringstyle=\ttfamily\color{purple},
mathescape=false,escapechar=\#,
numbers=left, numberstyle=\scriptsize\color{gray}\ttfamily, language=ML, showspaces=false,showstringspaces=false,xleftmargin=15pt, 
morekeywords={string, float, int, bool},
classoffset=0,belowskip=\smallskipamount, aboveskip=\smallskipamount,
moredelim=**[is][\color{red}]{SSTR}{ESTR}
}
\usepackage[skip=2\baselineskip]{caption}
\newcommand{\li}[1]{\lstinline[basicstyle=\ttfamily\fontsize{9pt}{1em}\selectfont]{#1}}
\newcommand{\lismall}[1]{\lstinline[basicstyle=\ttfamily\fontsize{9pt}{1em}\selectfont]{#1}}

%% Joshua Dunfield macros
\def\OPTIONConf{1}%
\usepackage{joshuadunfield}

%% Can remove this eventually
\usepackage{blindtext}

\usepackage{enumitem}

%%%%%%%%%%%%%%%%%%%%%%%%%%%%%%%%%%%%%%%%%%%%%%%%%%%%%%%%%%%%%%%%%%%%%%%%%%%%%
%% Matt says: Cyrus, this package `adjustbox` seems directly related
%% to the `clipbox` error; To get rid of the error, I moved it last
%% (after other usepackages) and I added the line just above it, which
%% permits it to redefine `clipbox` (apparently also defined in
%% `pstricks`, and due to latex's complete lack of namespace
%% management, these would otherwise names clash).
\let\clipbox\relax
\usepackage[export]{adjustbox}% http://ctan.org/pkg/adjustbox
%%%%%%%%%%%%%%%%%%%%%%%%%%%%%%%%%%%%%%%%%%%%%%%%%%%%%%%%%%%%%%%%%%%%%%%%%%%%%%%%%


%%%%%%%%%%%%%%%%%%%%%%%%%%%%%%%%%%%%%%%%%%%%%%%%%%%%%%%%%%%%%%%%%%%%%%%%%%%%%%%%%
%\usepackage{draftwatermark}
%\SetWatermarkText{DRAFT}
%\SetWatermarkScale{1}
%%%%%%%%%%%%%%%%%%%%%%%%%%%%%%%%%%%%%%%%%%%%%%%%%%%%%%%%%%%%%%%%%%%%%%%%%%%%%%%%%


% A macro for the name of the system being described by ``this paper''
\newcommand{\HazelnutLive}{\textsf{Hazelnut Live}\xspace}
\newcommand{\Hazelnut}{\textsf{Hazelnut}\xspace}
% The mockup, work-in-progress system.
\newcommand{\Hazel}{\textsf{Hazel}\xspace}

% \newtheorem{theorem}{Theorem}[chapter]
% \newtheorem{lemma}[theorem]{Lemma}
% \newtheorem{corollary}[theorem]{Corollary}
% \newtheorem{definition}[theorem]{Definition}
% \newtheorem{assumption}[theorem]{Assumption}
% \newtheorem{condition}[theorem]{Condition}

\newtheoremstyle{slplain}% name
  {.15\baselineskip\@plus.1\baselineskip\@minus.1\baselineskip}% Space above
  {.15\baselineskip\@plus.1\baselineskip\@minus.1\baselineskip}% Space below
  {\slshape}% Body font
  {\parindent}%Indent amount (empty = no indent, \parindent = para indent)
  {\bfseries}%  Thm head font
  {.}%       Punctuation after thm head
  { }%      Space after thm head: " " = normal interword space;
        %       \newline = linebreak
  {}%       Thm head spec
\theoremstyle{slplain}
\newtheorem{thm}{Theorem}  % Numbered with the equation counter
\numberwithin{thm}{section}
\newtheorem{defn}[thm]{Definition}
\newtheorem{lem}[thm]{Lemma}
\newtheorem{prop}[thm]{Proposition}
\newtheorem{corol}[thm]{Corollary}
% \newtheorem{cor}[section]{Corollary}     
% \newtheorem{lem}[section]{Lemma}         
% \newtheorem{prop}[section]{Proposition}  

% \setlength{\abovedisplayskip}{0pt}
% \setlength{\belowdisplayskip}{0pt}
% \setlength{\abovedisplayshortskip}{0pt}
% \setlength{\belowdisplayshortskip}{0pt}



\makeatletter\if@ACM@journal\makeatother
%% Journal information (used by PACMPL format)
%% Supplied to authors by publisher for camera-ready submission
\acmJournal{PACMPL}
\acmVolume{1}
\acmNumber{1}
\acmArticle{1}
\acmYear{2018}
\acmMonth{3}
\acmDOI{10.1145/nnnnnnn.nnnnnnn}
\startPage{1}
\else\makeatother
%% Conference information (used by SIGPLAN proceedings format)
%% Supplied to authors by publisher for camera-ready submission
% \acmConference[]{ACM SIGPLAN Conference on Programming Languages}{January 01--03, 2017}{New York, NY, USA}

\acmYear{2018}
\acmISBN{978-x-xxxx-xxxx-x/YY/MM}
\acmDOI{10.1145/nnnnnnn.nnnnnnn}
\startPage{1}
\fi


%% Copyright information
%% Supplied to authors (based on authors' rights management selection;
%% see authors.acm.org) by publisher for camera-ready submission
\setcopyright{none}             %% For review submission
%\setcopyright{acmcopyright}
%\setcopyright{acmlicensed}
%\setcopyright{rightsretained}
%\copyrightyear{2017}           %% If different from \acmYear


% \fancyfoot{} % suppresses the footer (also need \thispagestyle{empty} after \maketitle below)


%% Bibliography style
\bibliographystyle{ACM-Reference-Format}
%% Citation style
%% Note: author/year citations are required for papers published as an
%% issue of PACMPL.
\citestyle{acmauthoryear}   %% For author/year citations

% !TEX root = ./patterns-paper.tex

\newcommand{\mynote}[3]{\textcolor{#3}{\textsf{{#2}}}}
\newcommand{\rkc}[1]{\mynote{rkc}{#1}{blue}}
\newcommand{\cy}[1]{\mynote{cy}{#1}{purple}}
\newcommand{\mah}[1]{\mynote{cy}{#1}{green}}
\newcommand{\matt}[1]{{\color{blue}{\textit{Matt:~#1}}}}

\newcommand{\cvert}{{\,{\vert}\,}}

%% https://tex.stackexchange.com/questions/9796/how-to-add-todo-notes
\newcommand{\rkcTodo}[1]{\todo[linecolor=blue,backgroundcolor=blue!25,bordercolor=blue]{#1}}

\newcommand{\mattTodo}[1]{\todo[linecolor=green,backgroundcolor=green!2,bordercolor=green]{\tiny\textit{#1}}}
\newcommand{\mattOmit}[1]{\colorbox{yellow}{(Matt omitted stuff here)}}

\def\parahead#1{\paragraph{\textbf{#1.}}}
%% \def\paraheadNoDot#1{\paragraph{{\textbf{#1}}}}
\def\subparahead#1{\paragraph{\textit{#1.}}}
%% \def\paraheadindent#1{\paragraph{}\textit{#1.}}
%% \def\paraheadindentnodot#1{\paragraph{}\textit{#1}}

% \newcommand{\ie}{{\emph{i.e.}}}
% \newcommand{\eg}{{\emph{e.g.}}}
% \newcommand{\etc}{{\emph{etc.}}}
% \newcommand{\cf}{{\emph{cf.}}}
% \newcommand{\etal}{{\emph{et al.}}}

%% \newcommand{\hazel}{\ensuremath{\textsc{Hazel}}}
%% \newcommand{\sns}{\ensuremath{\textsc{Sketch-n-Sketch}}}
%% \newcommand{\deuce}{\ensuremath{\textsc{Deuce}}}
\newcommand{\Elm}{\ensuremath{\textsf{Elm}}}
\newcommand{\sns}{\ensuremath{\textrm{Sketch-n-Sketch}}}
\newcommand{\deuce}{\ensuremath{\textrm{Deuce}}}

\newcommand{\sectionDescription}[1]{\section{#1}}
\newcommand{\subsectionDescription}[1]{\subsection{#1}}
\newcommand{\subsubsectionDescription}[1]{\subsubsection{#1}}
%% \newcommand{\subsectionDescription}[1]{\subsection*{#1}}
\newcommand{\suppMaterials}{the Supplementary Materials}

\newcommand{\defeq}{\overset{\textrm{def}}{=}}

\newcommand{\eap}{action suggestion panel\xspace}
\newcommand{\Eap}{Action suggestion panel\xspace}

\newcommand{\myfootnote}[1]{\footnote{ #1}}

\def\sectionautorefname{Section}
\def\subsectionautorefname{Section}
\def\subsubsectionautorefname{Section}

\newcommand{\code}[1]{\lstinline{#1}}

% Make italic?
%\newcommand{\Property}[1]{\emph{#1}}
\newcommand{\Property}[1]{\textrm{#1}}

% Calling out Cyrus's favorite verb, 'to be' ;)
\newcommand{\IS}{\colorbox{red}{is}\xspace}

\newcommand{\codeSize}
  %% {\footnotesize}
  {\small}

%\newcommand{\JoinTypes}[2]{\textsf{join}~~#1~~#2}
\newcommand{\JoinTypes}[2]{\textsf{join}(#1,#2)}

%%%%%%%%%%%%%%%%%%%%%%%%%%%%%%%%%%%%%%%%%%%%%%%%%%%%%%%%%%%%%%%%%%%%%%%%%%%%%%%%
%% Spacing

\newcommand{\sep}{\hspace{0.06in}}
\newcommand{\sepPremise}{\hspace{0.20in}}
\newcommand{\hsepRule}{\hspace{0.20in}}
\newcommand{\vsepRuleHeight}{0.08in}
\newcommand{\vsepRule}{\vspace{\vsepRuleHeight}}
\newcommand{\miniSepOne}{\hspace{0.01in}}
\newcommand{\miniSepTwo}{\hspace{0.02in}}
\newcommand{\miniSepThree}{\hspace{0.03in}}
\newcommand{\miniSepFour}{\hspace{0.04in}}
\newcommand{\miniSepFive}{\hspace{0.05in}}

%%%%%%%%%%%%%%%%%%%%%%%%%%%%%%%%%%%%%%%%%%%%%%%%%%%%%%%%%%%%%%%%%%%%%%%%%%%%%%%%

% \lstset{
% %mathescape=true,basicstyle=\fontsize{8}{9}\ttfamily,
% literate={=>}{$\Rightarrow$}2
%          {<=}{$\leq$}2
%          {->}{${\rightarrow}$}1
%          {\\\\=}{\color{red}{$\lambda$}}2
%          {\\\\}{$\lambda$}2
%          {**}{$\times$}2
%          {*.}{${\color{blue}{\texttt{*.}}}$}2
%          {+.}{${\color{blue}{\texttt{+.}}}$}2
%          {<}{${\color{green}{\lhd}}$}1
%          {>?}{${\color{green}{\rhd}}$?}2
%          {<<}{${\color{green}{\blacktriangleleft}}$}1
%          {>>?}{${\color{green}{\blacktriangleright}}$?}2
%          {\{}{${\color{blue}{\{}}$}1
%          {\}}{${\color{blue}{\}}}$}1
%          {[}{${\color{purple}{[}}$}1
%          {]}{${\color{purple}{]}}$}1
%          {(}{${\color{darkgray}{\texttt{(}}}$}1
%          {)}{${\color{darkgray}{\texttt{)}}}$}1
%          {]]}{${\color{gray}{\big(}}$}1
%          {]]}{${\color{gray}{\big)}}$}1
% }

% !TEX root = ./patterns-paper.tex

% reverse Vdash
\newcommand{\dashV}{\mathbin{\rotatebox[origin=c]{180}{$\Vdash$}}}

% Violet hotdogs; highlight color helps distinguish them
\newcommand{\llparenthesiscolor}{\textcolor{violet}{\llparenthesis}}
\newcommand{\rrparenthesiscolor}{\textcolor{violet}{\rrparenthesis}}

% HTyp and HExp
\newcommand{\hcomplete}[1]{#1~\mathsf{complete}}

% HTyp
\newcommand{\htau}{\dot{\tau}}
\newcommand{\tarr}[2]{\inparens{#1 \rightarrow #2}}
\newcommand{\tarrnp}[2]{#1 \rightarrow #2}
\newcommand{\trul}[2]{\inparens{#1 \Rightarrow #2}}
\newcommand{\trulnp}[2]{#1 \Rightarrow #2}
\newcommand{\tnum}{\mathtt{num}}
\newcommand{\tehole}{\llparenthesiscolor\rrparenthesiscolor}
\newcommand{\tsum}[2]{\inparens{{#1} + {#2}}}
\newcommand{\tprod}[2]{\inparens{{#1} \times {#2}}}
\newcommand{\tunit}{\mathtt{1}}
\newcommand{\tvoid}{\mathtt{0}}

\newcommand{\tcompat}[2]{#1 \sim #2}
\newcommand{\tincompat}[2]{#1 \nsim #2}

% HExp
\newcommand{\hexp}{\dot{e}}
\newcommand{\hlam}[3]{\inparens{\lambda #1:#2.#3}}
\newcommand{\hap}[2]{#1(#2)}
\newcommand{\hapP}[2]{(#1)~(#2)} % Extra paren around function term
\newcommand{\hnum}[1]{\underline{#1}}
\newcommand{\hadd}[2]{\inparens{#1 + #2}}
\newcommand{\hpair}[2]{\inparens{#1 , #2}}
\newcommand{\htriv}{()}
\newcommand{\hehole}[1]{\llparenthesiscolor\rrparenthesiscolor^{#1}}
\newcommand{\hhole}[2]{\llparenthesiscolor#1\rrparenthesiscolor^{#2}}
\newcommand{\hindet}[1]{\lceil#1\rceil}
\newcommand{\hinj}[2]{\mathtt{inj}_{#1}({#2})}
\newcommand{\hinl}[2]{\mathtt{inl}_{#1}({#2})}
\newcommand{\hinr}[2]{\mathtt{inr}_{#1}({#2})}
\newcommand{\hinlp}[1]{\mathtt{inl}(#1)}
\newcommand{\hinrp}[1]{\mathtt{inr}(#1)}
\newcommand{\hmatch}[2]{\mathtt{match}(#1) \{#2\}}
\newcommand{\hcase}[5]{\mathtt{case}({#1},{#2}.{#3},{#4}.{#5})}
\newcommand{\hrules}[2]{#1 \mid #2}
\newcommand{\hrulesP}[2]{\inparens{#1 \mid #2}}
\newcommand{\hrul}[2]{#1 \Rightarrow #2}
\newcommand{\hrulP}[2]{\inparens{#1 \Rightarrow #2}}

\newcommand{\hGamma}{\dot{\Gamma}}
\newcommand{\domof}[1]{\text{dom}(#1)}
\newcommand{\hsyn}[3]{#1 \vdash #2 \Rightarrow #3}
\newcommand{\hana}[3]{#1 \vdash #2 \Leftarrow #3}
\newcommand{\hexptyp}[4]{#1 \mathbin{;} #2 \vdash #3 : #4}
\newcommand{\hpattyp}[4]{#1 : #2 \dashV #3 \mathbin{;} #4}
\newcommand{\hsubstyp}[2]{#1 : #2}
\newcommand{\hpatmatch}[3]{#1 \vartriangleright #2 \dashV #3}
\newcommand{\hnotmatch}[2]{#1 \mathbin{\bot} #2}
\newcommand{\hmaymatch}[2]{#1 \mathbin{?} #2}
\newcommand{\htrans}[2]{#1 \mapsto #2}

\newcommand{\isVal}[1]{#1 ~\mathtt{val}}
\newcommand{\isErr}[1]{#1 ~\mathtt{err}}
\newcommand{\isIndet}[1]{#1 ~\mathtt{indet}}
\newcommand{\isFinal}[1]{#1 ~\mathtt{final}}

% ZTyp and ZExp
\newcommand{\zlsel}[1]{{\bowtie}{#1}}
\newcommand{\zrsel}[1]{{#1}{\bowtie}}
\newcommand{\zwsel}[1]{
  \setlength{\fboxsep}{0pt}
  \colorbox{green!10!white!100}{
    \ensuremath{{{\textcolor{Green}{{\hspace{-2px}\triangleright}}}}{#1}{\textcolor{Green}{\triangleleft{\vphantom{\tehole}}}}}}
}

\newcommand{\removeSel}[1]{#1^{\diamond}}

% ZTyp
\newcommand{\ztau}{\hat{\tau}}

% ZExp
\newcommand{\zexp}{\hat{e}}

% Direction
\newcommand{\dParent}{\mathtt{parent}}
\newcommand{\dChildn}[1]{\mathtt{child}~\mathtt{{#1}}}
\newcommand{\dChildnm}[1]{\mathtt{child}~{#1}}

% Action
\newcommand{\aMove}[1]{\mathtt{move}~#1}
	\newcommand{\zrightmost}[1]{\mathsf{rightmost}(#1)}
	\newcommand{\zleftmost}[1]{\mathsf{leftmost}(#1)}
\newcommand{\aSelect}[1]{\mathtt{sel}~#1}
\newcommand{\aDel}{\mathtt{del}}
\newcommand{\aReplace}[1]{\mathtt{replace}~#1}
\newcommand{\aConstruct}[1]{\mathtt{construct}~#1}
\newcommand{\aConstructx}[1]{#1}
\newcommand{\aFinish}{\mathtt{finish}}

\newcommand{\performAna}[5]{#1 \vdash #2 \xlongrightarrow{#4} #5 \Leftarrow #3}
\newcommand{\performAnaI}[5]{#1 \vdash #2 \xlongrightarrow{#4}\hspace{-3px}{}^{*}~ #5 \Leftarrow #3}
\newcommand{\performSyn}[6]{#1 \vdash #2 \Rightarrow #3 \xlongrightarrow{#4} #5 \Rightarrow #6}
\newcommand{\performSynI}[6]{#1 \vdash #2 \Rightarrow #3 \xlongrightarrow{#4}\hspace{-3px}{}^{*}~ #5 \Rightarrow #6}
\newcommand{\performTyp}[3]{#1 \xlongrightarrow{#2} #3}
\newcommand{\performTypI}[3]{#1 \xlongrightarrow{#2}\hspace{-3px}{}^{*}~#3}

\newcommand{\performMove}[3]{#1 \xlongrightarrow{#2} #3}
\newcommand{\performDel}[2]{#1 \xlongrightarrow{\aDel} #2}

% Form
\newcommand{\farr}{\mathtt{arrow}}
\newcommand{\fnum}{\mathtt{num}}
\newcommand{\fsum}{\mathtt{sum}}

\newcommand{\fasc}{\mathtt{asc}}
\newcommand{\fvar}[1]{\mathtt{var}~#1}
\newcommand{\flam}[1]{\mathtt{lam}~#1}
\newcommand{\fap}{\mathtt{ap}}
% \newcommand{\farg}{\mathtt{arg}}
\newcommand{\fnumlit}[1]{\mathtt{lit}~#1}
\newcommand{\fplus}{\mathtt{plus}}
\newcommand{\fhole}{\mathtt{hole}}
\newcommand{\fnehole}{\mathtt{nehole}}

\newcommand{\finj}[1]{\mathtt{inj}~#1}
\newcommand{\fcase}[2]{\mathtt{case}~#1~#2}

% Talk about formal rules in example
\newcommand{\refrule}[1]{\textrm{Rule~(#1)}}

\newcommand{\herase}[1]{\left|#1\right|_\textsf{erase}}

\newcommand{\arrmatch}[2]{#1 \blacktriangleright_{\rightarrow} #2}


\newcommand{\TABperformAna}[5]{#1 \vdash & #2                & \xlongrightarrow{#4} & #5 & \Leftarrow #3}
\newcommand{\TABperformSyn}[6]{#1 \vdash & #2 \Rightarrow #3 & \xlongrightarrow{#4} & #5 \Rightarrow #6}
\newcommand{\TABperformTyp}[3]{& #1 & \xlongrightarrow{#2} & #3}

\newcommand{\TABperformMove}[3]{#1 & \xlongrightarrow{#2} & #3}
\newcommand{\TABperformDel}[2]{#1 \xlongrightarrow{\aDel} #2}

\newcommand{\sumhasmatched}[2]{#1 \mathrel{\textcolor{black}{\blacktriangleright_{+}}} #2}

\newcommand{\subminsyn}[1]{\mathsf{submin}_{\Rightarrow}(#1)}
\newcommand{\subminana}[1]{\mathsf{submin}_{\Leftarrow}(#1)}


\newcommand{\inparens}[1]{{\color{gray}(}#1{\color{gray})}}

%% rule names for appendix
\newcommand{\rname}[1]{\textsc{#1}}
\newcommand{\gap}{\vspace{7pt}}


\setlength{\abovecaptionskip}{4pt plus 3pt minus 2pt} % Chosen fairly arbitrarily
\setlength{\belowcaptionskip}{-4pt plus 3pt minus 2pt} % Chosen fairly arbitrarily


\begin{document}

%% Title information
\title{Live and Direct Functional Programming with Holes}         %% [Short Title] is optional;
                                        %% when present, will be used in
                                        %% header instead of Full Title.
\titlenote{Some text and figures in this submission are taken from a full paper by the authors, which is currently under review.}
%% \titlenote is optional;
                                        %% can be repeated if necessary;
                                        %% contents suppressed with 'anonymous'
% \subtitle{Subtitle}                     %% \subtitle is optional
% \subtitlenote{with subtitle note}       %% \subtitlenote is optional;
                                        %% can be repeated if necessary;
                                        %% contents suppressed with 'anonymous'


%% Author information
%% Contents and number of authors suppressed with 'anonymous'.
%% Each author should be introduced by \author, followed by
%% \authornote (optional), \orcid (optional), \affiliation, and
%% \email.
%% An author may have multiple affiliations and/or emails; repeat the
%% appropriate command.
%% Many elements are not rendered, but should be provided for metadata
%% extraction tools.

%% Author with single affiliation.
\author{Cyrus Omar}
% \authornote{with author1 note}          %% \authornote is optional;
                                        %% can be repeated if necessary
% \orcid{nnnn-nnnn-nnnn-nnnn}             %% \orcid is optional
\affiliation{
  % \position{Position1}
  % \department{Department1}              %% \department is recommended
  \institution{University of Chicago}            %% \institution is required
  % \streetaddress{Street1 Address1}
  % \city{City1}
  % \state{State1}
  % \postcode{Post-Code1}
  % \country{Country1}
}
\email{comar@cs.uchicago.edu}          %% \email is recommended

\author{Ian Voysey}
% \authornote{with author1 note}          %% \authornote is optional;
                                        %% can be repeated if necessary
% \orcid{nnnn-nnnn-nnnn-nnnn}             %% \orcid is optional
\affiliation{
  % \position{Position1}
  % \department{Department1}              %% \department is recommended
  \institution{Carnegie Mellon University}            %% \institution is required
  % \streetaddress{Street1 Address1}
  % \city{City1}
  % \state{State1}
  % \postcode{Post-Code1}
  % \country{Country1}
}
\email{iev@cs.cmu.edu}          %% \email is recommended

\author{Ravi Chugh}
% \authornote{with author1 note}          %% \authornote is optional;
                                        %% can be repeated if necessary
% \orcid{nnnn-nnnn-nnnn-nnnn}             %% \orcid is optional
\affiliation{
  % \position{Position1}
  % \department{Department1}              %% \department is recommended
  \institution{University of Chicago}            %% \institution is required
  % \streetaddress{Street1 Address1}
  % \city{City1}
  % \state{State1}
  % \postcode{Post-Code1}
  % \country{Country1}
}
\email{rchugh@cs.uchicago.edu}          %% \email is recommended

\author{Matthew A. Hammer}
% \authornote{with author1 note}          %% \authornote is optional;
                                        %% can be repeated if necessary
% \orcid{nnnn-nnnn-nnnn-nnnn}             %% \orcid is optional
\affiliation{
  % \position{Position1}
  % \department{Department1}              %% \department is recommended
  \institution{University of Colorado Boulder}            %% \institution is required
  % \streetaddress{Street1 Address1}
  % \city{City1}
  % \state{State1}
  % \postcode{Post-Code1}
  % \country{Country1}
}
\email{matthew.hammer@colorado.edu}          %% \email is recommended


% %% Author with two affiliations and emails.
% \author{First2 Last2}
% \authornote{with author2 note}          %% \authornote is optional;
%                                         %% can be repeated if necessary
% \orcid{nnnn-nnnn-nnnn-nnnn}             %% \orcid is optional
% \affiliation{
%   \position{Position2a}
%   \department{Department2a}             %% \department is recommended
%   \institution{Institution2a}           %% \institution is required
%   \streetaddress{Street2a Address2a}
%   \city{City2a}
%   \state{State2a}
%   \postcode{Post-Code2a}
%   \country{Country2a}
% }
% \email{first2.last2@inst2a.com}         %% \email is recommended
% \affiliation{
%   \position{Position2b}
%   \department{Department2b}             %% \department is recommended
%   \institution{Institution2b}           %% \institution is required
%   \streetaddress{Street3b Address2b}
%   \city{City2b}
%   \state{State2b}
%   \postcode{Post-Code2b}
%   \country{Country2b}
% }
% \email{first2.last2@inst2b.org}         %% \email is recommended


%% Paper note
%% The \thanks command may be used to create a "paper note" ---
%% similar to a title note or an author note, but not explicitly
%% associated with a particular element.  It will appear immediately
%% above the permission/copyright statement.
% \thanks{with paper note}                %% \thanks is optional
                                        %% can be repeated if necesary
                                        %% contents suppressed with 'anonymous'


%% Abstract
%% Note: \begin{abstract}...\end{abstract} environment must come
%% before \maketitle command
% !TEX root = hazel-LIVE2018.tex

\begin{abstract}
\emph{Type inference} allows programmers to omit type annotations while still permitting type checking. The problem is that \emph{type inference} only works on complete programs, e.g. syntactically well-formed expressions. This paper introduces a \emph{type hole inference} system that applies type inference on incomplete programs. We start from modeling incomplete programs by \emph{holes} developed in \HazelnutLive, where holes stand for missing types and expressions. Rather than obtaining limited static type information for type holes from bidirectional typing propagation, we extended static semantics with \emph{type constraints}, and borrow machinery from unification-based type inference to infer types for type holes. Our approach shares benefits of \emph{bidirectional typing}, which produces good-quality error messages, and unification, which infers types for unknown parts of incomplete programs to collect as much static type information as possible. 


\end{abstract}
    


%% 2012 ACM Computing Classification System (CSS) concepts
%% Generate at 'http://dl.acm.org/ccs/ccs.cfm'.
% \begin{CCSXML}
% <ccs2012>
% <concept>
% <concept_id>10011007.10011006.10011008</concept_id>
% <concept_desc>Software and its engineering~General programming languages</concept_desc>
% <concept_significance>500</concept_significance>
% </concept>
% <concept>
% <concept_id>10003456.10003457.10003521.10003525</concept_id>
% <concept_desc>Social and professional topics~History of programming languages</concept_desc>
% <concept_significance>300</concept_significance>
% </concept>
% </ccs2012>
% \end{CCSXML}

% \ccsdesc[500]{Software and its engineering~General programming languages}
% \ccsdesc[300]{Social and professional topics~History of programming languages}
%% End of generated code


%% Keywords
%% comma separated list
% \keywords{keyword1, keyword2, keyword3}  %% \keywords is optional


%% \maketitle
%% Note: \maketitle command must come after title commands, author
%% commands, abstract environment, Computing Classification System
%% environment and commands, and keywords command.
\maketitle
% \thispagestyle{empty} % suppresses the footer

\section{Introduction}
\label{sec:intro}

\section{Type System for Type Hole Inference}
\label{sec:typinf}
Figure ~\ref{fig:syntax_fig} defines the syntax of H-types and H-expressions. We start from the definition given in \citet{HazelnutPOPL} except two modifications. First, we add an annotated lambda. Second, we attach a unique \emph{type hole identifier}, drawn as a superscript capital letter, to each type hole, such as $\ehole^A$. In this way, type holes are identified as type variables. There are two types of expression holes: empty holes, $\ehole$, standing for missing parts of an incomplete program, and non-empty holes, $\notehole{e}$, operating as membranes around static and dynamic type inconsistencies \cite{HazelLive}. \par
\begin{figure}[htbp]
\vspace{-3px} 
$\arraycolsep=4pt\begin{array}{lll}
HTyp~~ \tau & ::= &
  \tnum ~\vert~
  \tarr{\tau}{\tau} ~\vert~
  \ehole^A
  \\
HExp~~ e & ::= &
  x ~\vert~
  \lamfunc{x}{e} ~\vert~
  \lamfunc{x:\tau}{e} ~\vert~
  e(e) ~\vert~
  \underlinenum{n} ~\vert~
  (e+e) ~\vert~
  e: \tau ~\vert~
  \ehole  ~\vert~
  \notehole{e} 
\end{array}$
\hrule
\caption{Syntax of H-types, H-expressions, and Type Constraint Set}
\label{fig:syntax_fig}
\vspace{-5px}
\end{figure}
Figure ~\ref{fig:ana-syn} defines a bidirectional typing system extended with type constraint sets. \emph{Type constraint set} $C$ is a set of type consistency equations, namely \emph{type constraints}. Type consistency $\tau \sim \tau$ is a symmetric and reflective but not transitive relation defined in figure ~\ref{fig:type-consistency} \cite{HazelnutPOPL}. We use a union operation, $C \cup C$, corresponding to mathematical set union operation to generate constraints inductively through bidirectional propagation. The type constraint set is solved by the unification algorithm in section ~\ref{sec:infalg}, but importantly, constraint solving is not necessary for typing to succeed. The typing context, $\Gamma$, maps a set of expression variables to their types. Rule ~\ref{rule:syn-ehole} and ~\ref{rule:syn-hole} synthesize expression hole to hole type, with premise, $(A ~ \text{fresh})$, indicating generation of a new \emph{type identifier}. Rule ~\ref{rule:ana-subsume} and ~\ref{rule:ana-lam} have type consistency in their premises, generating new constraints and merging them into constraint sets in the conclusion. Rule ~\ref{rule:syn-ap}, ~\ref{rule:ana-lam} and ~\ref{rule:ana-lamann} have \emph{matched arrow type judgements} defined in figure ~\ref{fig:match-arrow-typ}. They leave the arrow type unchanged and assign the type hole the matched arrow type $\tarr{\tehole^B}{\tehole^C}$ with fresh identifiers and constraint generation \cite{HazelnutPOPL}.
\begin{figure}
\vspace{-3px} 
    \begin{multicols}{2}
      \fbox{$\consexptyp{\Gamma}{e}{\tau}{C}$}~~\text{$e$ synthesizes $\tau$}\hfill
    \begin{subequations}
    \begin{equation}\label{rule:syn-var}
        \inferrule[]{ }{
            \consexptyp{\Gamma, x : \tau}{x}{\tau}{\econs}
          }
    \end{equation}
    \begin{equation}\label{rule:syn-num}
        \inferrule[]{ }{
            \consexptyp{\Gamma}{\hnum{n}}{\tnum}{\econs}
          }
    \end{equation}
    \begin{equation}\label{rule:syn-plus}
        \inferrule[]{
            \ana{\Gamma}{e_1}{\tnum}{C_1} \\
            \ana{\Gamma}{e_2}{\tnum}{C_2}
          }{
            \consexptyp{\Gamma}{(e_1 + e_2)}{\tnum}{C_1 \cup C_2}
          }
    \end{equation}
    \begin{equation}\label{rule:syn-asc}
        \inferrule[]{
            \ana{\Gamma}{e}{\tau}{C}
          }{
            \consexptyp{\Gamma}{(e : \tau)}{\tau}{C}
          }
    \end{equation}
    \begin{equation}\label{rule:syn-ehole}
        \inferrule[]{(A~\text{fresh}) }{
            \consexptyp{\Gamma}{\llparenthesiscolor \rrparenthesiscolor}{\llparenthesiscolor \rrparenthesiscolor^A}{\econs}
          }
    \end{equation}
    \begin{equation}\label{rule:syn-hole}
        \inferrule[]{
            (A~\text{fresh}) \\
            \consexptyp{\Gamma}{e}{\tau}{C}
           }{
             \consexptyp{\Gamma}{\llparenthesiscolor e \rrparenthesiscolor}{\llparenthesiscolor \rrparenthesiscolor^A}{C}
           }
    \end{equation}
    \begin{equation}\label{rule:syn-lamann}
        \inferrule[]{
          \consexptyp{\Gamma, x : \tau_{in}}{e}{\tau_{out}}{C}
        }{
          \consexptyp{\Gamma}{\lamfunc{x:\tau_{in}}{e}}{\tarr{\tau_{in}}{\tau_{out}}}{C}
        }
    \end{equation}
    \begin{equation}\label{rule:syn-ap}
      \inferrule[]{
          \consexptyp{\Gamma}{e_1}{\tau_1}{C_1} \\
          \tau_1 \typearrow \tarr{\tau_{in}}{\tau_{out}} \addcons{C_2} \\
          \ana{\Gamma}{e_2}{\tau_{in}}{C_3}
        }{
          \consexptyp{\Gamma}{\hap{e_1}{e_2}}{\tau_{out}} { C_1 \cup C_2 \cup C_3}
        }
  \end{equation}
    \end{subequations}
    \vspace{3px}\fbox{$\ana{\Gamma}{e}{\tau} {C}$}~~\text{$e$ analyzes against $\tau$}\hfill
    \begin{subequations}
    \begin{equation}\label{rule:ana-subsume}
      \inferrule[]{
          \consexptyp{\Gamma}{e}{\tau'}{C_1} \\
          \tau \sim \tau' 
        }{
          \ana{\Gamma}{e}{\tau}{C_1 \cup \{\tau \sim \tau'  \}}
        }
  \end{equation}
    \begin{equation}\label{rule:ana-lam}
        \inferrule[]{
            \tau \typearrow \tarr{\tau_{in}}{\tau_{out}} \addcons{C_1} \\
             \ana{\Gamma, x : \tau_{in}}{e}{\tau_{out}}{C_2}
           }{
             \ana{\Gamma}{\lamfunc{x}{e}}{\tau}{C_1 \cup C_2}
           }
    \end{equation}
    \begin{equation}\label{rule:ana-lamann}
        \inferrule[]{
         \tau \typearrow \tarr{\tau_{in}}{\tau_{out}} \addcons{C_1} \\
          \ana{\Gamma, x : \tau'_{in}}{e}{\tau_{out}}{C_2} \\
          \tau_{in} \sim \tau'_{in}
        }{
          \ana{\Gamma}{\lamfunc{x:\tau'_{in}}{e}}{\tau}{C_1 \cup C_2 \cup \{ \tau_{in} \sim \tau'_{in} \}}
        }
    \end{equation}
    \end{subequations}
  \end{multicols}
  \hrule
  \caption{H-type synthesis and analysis.}
  \label{fig:ana-syn}
  \vspace{-10px}
\end{figure}

\begin{figure}
   \fbox{$\tau \sim \tau $}~~\text{$\tau$ is consistent to $\tau$}\hfill
    \begin{subequations}\label{eqns:consistency}
    \begin{mathpar}
      \hfill
        \inferrule[]{
            }{
              \tnum \sim \tnum
            }
            \hfill
    \inferrule[]{
        }{
        \tehole^A \sim \tau
        }
        \hfill
    \inferrule[]{
        }{
        \tau \sim \tehole^A
        }
        \hfill
    \inferrule[]{
        \tau_1 \sim \tau_3 \\
        \tau_2 \sim \tau_4
        }{
        \tarr{\tau_1}{\tau_2} \sim \tarr{\tau_3}{\tau_4}
        }\hfill \text{\hspace{-2px}(\ref*{eqns:consistency}a-d)}
    \end{mathpar}
  \end{subequations}
  \hrule
  \caption{H-type consistency.}
  \label{fig:type-consistency}
  \vspace{-3px}
\end{figure}

\begin{figure}
    \fbox{$\tau \typearrow \tarr{\tau_{in}}{\tau_{out}} \addcons{C}$}~~\text{$\tau$ has matching arrow type $\tarr{\tau_{in}}{\tau_{out}}$}\hfill
    \begin{subequations}\label{eqns:matcharrow}
      \begin{minipage}{0.43\linewidth}
        \begin{equation}
          \inferrule[]{ }{
            \tarr{\tau_{in}}{\tau_{out}} \typearrow \tarr{\tau_{in}}{\tau_{out}} \addcons{\econs}
          }
        \end{equation}
        \end{minipage}
        \begin{minipage}{0.55\linewidth}
        \begin{equation}
          \inferrule[ ]{
            (B~\text{fresh}) \\
            (C~\text{fresh})
           }{
             \tehole^A \typearrow \tarr{\tehole^B}{\tehole^C} \addcons{\{ \tehole^A \sim \tarr{\tehole^B}{\tehole^C} \} }
           }
        \end{equation}
        \end{minipage}

    \end{subequations}
    \hrule
    \caption{Matched arrow types.}
    \label{fig:match-arrow-typ} 
    \vspace{-2px} 
  \end{figure}

\section{Inference Algorithm}
\label{sec:infalg}
\emph{Type hole inference} takes two steps. First, use bidirectional static semantics extended with holes for type checking, \emph{local type inference} and type constraints collecting. \emph{Type constraints} are type consistency equations appearing in the premise of static semantics rules. The system triggers a static error if a program fails bidirectional checking, otherwise, move to the next step. Second, solve type constraints with a unification-based algorithm to infer types for type holes. If type constraints conflict in unification, the system postpones remaining type checking to runtime. 
\clearpage
%\bibliography{references,all.short,hazel_NSF}

\end{document}
