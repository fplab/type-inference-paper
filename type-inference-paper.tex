\PassOptionsToPackage{svgnames,dvipsnames,svgnames}{xcolor}
\newif\ifarxiv
\arxivtrue
% \arxivfalse
\ifarxiv
\documentclass[acmsmall,screen,nonacm]{acmart}
\else
%%% Note: arxiv does not want line numbers (they are detected somehow, and are not allowed)
\documentclass[acmsmall,screen]{acmart}
% \settopmatter{printfolios=false,printccs=false,printacmref=false}
% \settopmatter{printccs=false,printacmref=false}
\fi

\ifarxiv
\newcommand{\appendixName}{appendix}
\else
\newcommand{\appendixName}{extended appendix}
\fi

%% For single-blind review submission
%\documentclass[acmlarge,review]{acmart}\settopmatter{printfolios=true}
%% For final camera-ready submission
%\documentclass[acmlarge]{acmart}\settopmatter{}

%% Note: Authors migrating a paper from PACMPL format to traditional
%% SIGPLAN proceedings format should change 'acmlarge' to
%% 'sigplan,10pt'.

% \bibliographystyle{ACM-Reference-Format}


%% Some recommended packages.
\usepackage{booktabs}   %% For formal tables:
                        %% http://ctan.org/pkg/booktabs
\usepackage{subcaption} %% For complex figures with subfigures/subcaptions
                        %% http://ctan.org/pkg/subcaption

%% Cyrus packages
\usepackage{microtype}
\usepackage{mdframed}
\usepackage{colortab}
\usepackage{mathpartir}
\usepackage{enumitem}
\usepackage{bbm}
\usepackage{stmaryrd}
\usepackage{mathtools}
\usepackage{leftidx}
\usepackage{todonotes}
\usepackage{xspace}
\usepackage{wrapfig}
\usepackage{extarrows}
% \usepackage[subtle]{savetrees}

\usepackage{listings}%
\lstloadlanguages{ML}
\lstset{tabsize=2, 
basicstyle=\footnotesize\ttfamily, 
% keywordstyle=\sffamily,
commentstyle=\itshape\ttfamily\color{gray}, 
stringstyle=\ttfamily\color{purple},
mathescape=false,escapechar=\#,
numbers=left, numberstyle=\scriptsize\color{gray}\ttfamily, language=ML, showspaces=false,showstringspaces=false,xleftmargin=15pt, 
morekeywords={string, float, int, bool},
classoffset=0,belowskip=\smallskipamount, aboveskip=\smallskipamount,
moredelim=**[is][\color{red}]{SSTR}{ESTR}
}
\newcommand{\li}[1]{\lstinline[basicstyle=\ttfamily\fontsize{9pt}{1em}\selectfont]{#1}}
\newcommand{\lismall}[1]{\lstinline[basicstyle=\ttfamily\fontsize{9pt}{1em}\selectfont]{#1}}

%% Joshua Dunfield macros
\def\OPTIONConf{1}%
\usepackage{joshuadunfield}

%% Can remove this eventually
\usepackage{blindtext}

\usepackage{enumitem}

%%%%%%%%%%%%%%%%%%%%%%%%%%%%%%%%%%%%%%%%%%%%%%%%%%%%%%%%%%%%%%%%%%%%%%%%%%%%%
%% Matt says: Cyrus, this package `adjustbox` seems directly related
%% to the `clipbox` error; To get rid of the error, I moved it last
%% (after other usepackages) and I added the line just above it, which
%% permits it to redefine `clipbox` (apparently also defined in
%% `pstricks`, and due to latex's complete lack of namespace
%% management, these would otherwise names clash).
\let\clipbox\relax
\usepackage[export]{adjustbox}% http://ctan.org/pkg/adjustbox
%%%%%%%%%%%%%%%%%%%%%%%%%%%%%%%%%%%%%%%%%%%%%%%%%%%%%%%%%%%%%%%%%%%%%%%%%%%%%%%%%


%%%%%%%%%%%%%%%%%%%%%%%%%%%%%%%%%%%%%%%%%%%%%%%%%%%%%%%%%%%%%%%%%%%%%%%%%%%%%%%%%
%\usepackage{draftwatermark}
%\SetWatermarkText{DRAFT}
%\SetWatermarkScale{1}
%%%%%%%%%%%%%%%%%%%%%%%%%%%%%%%%%%%%%%%%%%%%%%%%%%%%%%%%%%%%%%%%%%%%%%%%%%%%%%%%%


% A macro for the name of the system being described by ``this paper''
\newcommand{\HazelnutLive}{\textsf{Hazelnut Live}\xspace}
\newcommand{\Hazelnut}{\textsf{Hazelnut}\xspace}
% The mockup, work-in-progress system.
\newcommand{\Hazel}{\textsf{Hazel}\xspace}

% \newtheorem{theorem}{Theorem}[chapter]
% \newtheorem{lemma}[theorem]{Lemma}
% \newtheorem{corollary}[theorem]{Corollary}
% \newtheorem{definition}[theorem]{Definition}
% \newtheorem{assumption}[theorem]{Assumption}
% \newtheorem{condition}[theorem]{Condition}

\newtheoremstyle{slplain}% name
  {.15\baselineskip\@plus.1\baselineskip\@minus.1\baselineskip}% Space above
  {.15\baselineskip\@plus.1\baselineskip\@minus.1\baselineskip}% Space below
  {\slshape}% Body font
  {\parindent}%Indent amount (empty = no indent, \parindent = para indent)
  {\bfseries}%  Thm head font
  {.}%       Punctuation after thm head
  { }%      Space after thm head: " " = normal interword space;
        %       \newline = linebreak
  {}%       Thm head spec
\theoremstyle{slplain}
\newtheorem{thm}{Theorem}  % Numbered with the equation counter
\numberwithin{thm}{section}
\newtheorem{defn}[thm]{Definition}
\newtheorem{lem}[thm]{Lemma}
\newtheorem{prop}[thm]{Proposition}
\newtheorem{corol}[thm]{Corollary}
% \newtheorem{cor}[section]{Corollary}     
% \newtheorem{lem}[section]{Lemma}         
% \newtheorem{prop}[section]{Proposition}  

% \setlength{\abovedisplayskip}{0pt}
% \setlength{\belowdisplayskip}{0pt}
% \setlength{\abovedisplayshortskip}{0pt}
% \setlength{\belowdisplayshortskip}{0pt}


\ifarxiv
\setcopyright{rightsretained} 
\acmJournal{PACMPL}
\acmYear{2019} \acmVolume{3} \acmNumber{POPL} \ifarxiv \acmArticle{1} \else \acmArticle{14} \fi \acmMonth{1} \acmPrice{}\acmDOI{10.1145/3290327}
\copyrightyear{2019}
\else
%%% The following is specific to POPL '19 and the paper
%%% 'Live Functional Programming with Typed Holes'
%%% by Cyrus Omar, Ian Voysey, Ravi Chugh, and Matthew A. Hammer.
%%%
\setcopyright{rightsretained}
\acmPrice{}
\acmDOI{10.1145/3290327}
\acmYear{2019}
\copyrightyear{2019}
\acmJournal{PACMPL}
\acmVolume{3}
\acmNumber{POPL}
\acmArticle{14}
\acmMonth{1}
\fi


% \fancyfoot{} % suppresses the footer (also need \thispagestyle{empty} after \maketitle below)


%% Bibliography style
\bibliographystyle{ACM-Reference-Format}
%% Citation style
%% Note: author/year citations are required for papers published as an
%% issue of PACMPL.
\citestyle{acmauthoryear}   %% For author/year citations

% !TEX root = ./patterns-paper.tex

\newcommand{\mynote}[3]{\textcolor{#3}{\textsf{{#2}}}}
\newcommand{\rkc}[1]{\mynote{rkc}{#1}{blue}}
\newcommand{\cy}[1]{\mynote{cy}{#1}{purple}}
\newcommand{\mah}[1]{\mynote{cy}{#1}{green}}
\newcommand{\matt}[1]{{\color{blue}{\textit{Matt:~#1}}}}

\newcommand{\cvert}{{\,{\vert}\,}}

%% https://tex.stackexchange.com/questions/9796/how-to-add-todo-notes
\newcommand{\rkcTodo}[1]{\todo[linecolor=blue,backgroundcolor=blue!25,bordercolor=blue]{#1}}

\newcommand{\mattTodo}[1]{\todo[linecolor=green,backgroundcolor=green!2,bordercolor=green]{\tiny\textit{#1}}}
\newcommand{\mattOmit}[1]{\colorbox{yellow}{(Matt omitted stuff here)}}

\def\parahead#1{\paragraph{\textbf{#1.}}}
%% \def\paraheadNoDot#1{\paragraph{{\textbf{#1}}}}
\def\subparahead#1{\paragraph{\textit{#1.}}}
%% \def\paraheadindent#1{\paragraph{}\textit{#1.}}
%% \def\paraheadindentnodot#1{\paragraph{}\textit{#1}}

% \newcommand{\ie}{{\emph{i.e.}}}
% \newcommand{\eg}{{\emph{e.g.}}}
% \newcommand{\etc}{{\emph{etc.}}}
% \newcommand{\cf}{{\emph{cf.}}}
% \newcommand{\etal}{{\emph{et al.}}}

%% \newcommand{\hazel}{\ensuremath{\textsc{Hazel}}}
%% \newcommand{\sns}{\ensuremath{\textsc{Sketch-n-Sketch}}}
%% \newcommand{\deuce}{\ensuremath{\textsc{Deuce}}}
\newcommand{\Elm}{\ensuremath{\textsf{Elm}}}
\newcommand{\sns}{\ensuremath{\textrm{Sketch-n-Sketch}}}
\newcommand{\deuce}{\ensuremath{\textrm{Deuce}}}

\newcommand{\sectionDescription}[1]{\section{#1}}
\newcommand{\subsectionDescription}[1]{\subsection{#1}}
\newcommand{\subsubsectionDescription}[1]{\subsubsection{#1}}
%% \newcommand{\subsectionDescription}[1]{\subsection*{#1}}
\newcommand{\suppMaterials}{the Supplementary Materials}

\newcommand{\defeq}{\overset{\textrm{def}}{=}}

\newcommand{\eap}{action suggestion panel\xspace}
\newcommand{\Eap}{Action suggestion panel\xspace}

\newcommand{\myfootnote}[1]{\footnote{ #1}}

\def\sectionautorefname{Section}
\def\subsectionautorefname{Section}
\def\subsubsectionautorefname{Section}

\newcommand{\code}[1]{\lstinline{#1}}

% Make italic?
%\newcommand{\Property}[1]{\emph{#1}}
\newcommand{\Property}[1]{\textrm{#1}}

% Calling out Cyrus's favorite verb, 'to be' ;)
\newcommand{\IS}{\colorbox{red}{is}\xspace}

\newcommand{\codeSize}
  %% {\footnotesize}
  {\small}

%\newcommand{\JoinTypes}[2]{\textsf{join}~~#1~~#2}
\newcommand{\JoinTypes}[2]{\textsf{join}(#1,#2)}

%%%%%%%%%%%%%%%%%%%%%%%%%%%%%%%%%%%%%%%%%%%%%%%%%%%%%%%%%%%%%%%%%%%%%%%%%%%%%%%%
%% Spacing

\newcommand{\sep}{\hspace{0.06in}}
\newcommand{\sepPremise}{\hspace{0.20in}}
\newcommand{\hsepRule}{\hspace{0.20in}}
\newcommand{\vsepRuleHeight}{0.08in}
\newcommand{\vsepRule}{\vspace{\vsepRuleHeight}}
\newcommand{\miniSepOne}{\hspace{0.01in}}
\newcommand{\miniSepTwo}{\hspace{0.02in}}
\newcommand{\miniSepThree}{\hspace{0.03in}}
\newcommand{\miniSepFour}{\hspace{0.04in}}
\newcommand{\miniSepFive}{\hspace{0.05in}}

%%%%%%%%%%%%%%%%%%%%%%%%%%%%%%%%%%%%%%%%%%%%%%%%%%%%%%%%%%%%%%%%%%%%%%%%%%%%%%%%

% \lstset{
% %mathescape=true,basicstyle=\fontsize{8}{9}\ttfamily,
% literate={=>}{$\Rightarrow$}2
%          {<=}{$\leq$}2
%          {->}{${\rightarrow}$}1
%          {\\\\=}{\color{red}{$\lambda$}}2
%          {\\\\}{$\lambda$}2
%          {**}{$\times$}2
%          {*.}{${\color{blue}{\texttt{*.}}}$}2
%          {+.}{${\color{blue}{\texttt{+.}}}$}2
%          {<}{${\color{green}{\lhd}}$}1
%          {>?}{${\color{green}{\rhd}}$?}2
%          {<<}{${\color{green}{\blacktriangleleft}}$}1
%          {>>?}{${\color{green}{\blacktriangleright}}$?}2
%          {\{}{${\color{blue}{\{}}$}1
%          {\}}{${\color{blue}{\}}}$}1
%          {[}{${\color{purple}{[}}$}1
%          {]}{${\color{purple}{]}}$}1
%          {(}{${\color{darkgray}{\texttt{(}}}$}1
%          {)}{${\color{darkgray}{\texttt{)}}}$}1
%          {]]}{${\color{gray}{\big(}}$}1
%          {]]}{${\color{gray}{\big)}}$}1
% }

% !TEX root = ./patterns-paper.tex

% reverse Vdash
\newcommand{\dashV}{\mathbin{\rotatebox[origin=c]{180}{$\Vdash$}}}

% Violet hotdogs; highlight color helps distinguish them
\newcommand{\llparenthesiscolor}{\textcolor{violet}{\llparenthesis}}
\newcommand{\rrparenthesiscolor}{\textcolor{violet}{\rrparenthesis}}

% HTyp and HExp
\newcommand{\hcomplete}[1]{#1~\mathsf{complete}}

% HTyp
\newcommand{\htau}{\dot{\tau}}
\newcommand{\tarr}[2]{\inparens{#1 \rightarrow #2}}
\newcommand{\tarrnp}[2]{#1 \rightarrow #2}
\newcommand{\trul}[2]{\inparens{#1 \Rightarrow #2}}
\newcommand{\trulnp}[2]{#1 \Rightarrow #2}
\newcommand{\tnum}{\mathtt{num}}
\newcommand{\tehole}{\llparenthesiscolor\rrparenthesiscolor}
\newcommand{\tsum}[2]{\inparens{{#1} + {#2}}}
\newcommand{\tprod}[2]{\inparens{{#1} \times {#2}}}
\newcommand{\tunit}{\mathtt{1}}
\newcommand{\tvoid}{\mathtt{0}}

\newcommand{\tcompat}[2]{#1 \sim #2}
\newcommand{\tincompat}[2]{#1 \nsim #2}

% HExp
\newcommand{\hexp}{\dot{e}}
\newcommand{\hlam}[3]{\inparens{\lambda #1:#2.#3}}
\newcommand{\hap}[2]{#1(#2)}
\newcommand{\hapP}[2]{(#1)~(#2)} % Extra paren around function term
\newcommand{\hnum}[1]{\underline{#1}}
\newcommand{\hadd}[2]{\inparens{#1 + #2}}
\newcommand{\hpair}[2]{\inparens{#1 , #2}}
\newcommand{\htriv}{()}
\newcommand{\hehole}[1]{\llparenthesiscolor\rrparenthesiscolor^{#1}}
\newcommand{\hhole}[2]{\llparenthesiscolor#1\rrparenthesiscolor^{#2}}
\newcommand{\hindet}[1]{\lceil#1\rceil}
\newcommand{\hinj}[2]{\mathtt{inj}_{#1}({#2})}
\newcommand{\hinl}[2]{\mathtt{inl}_{#1}({#2})}
\newcommand{\hinr}[2]{\mathtt{inr}_{#1}({#2})}
\newcommand{\hinlp}[1]{\mathtt{inl}(#1)}
\newcommand{\hinrp}[1]{\mathtt{inr}(#1)}
\newcommand{\hmatch}[2]{\mathtt{match}(#1) \{#2\}}
\newcommand{\hcase}[5]{\mathtt{case}({#1},{#2}.{#3},{#4}.{#5})}
\newcommand{\hrules}[2]{#1 \mid #2}
\newcommand{\hrulesP}[2]{\inparens{#1 \mid #2}}
\newcommand{\hrul}[2]{#1 \Rightarrow #2}
\newcommand{\hrulP}[2]{\inparens{#1 \Rightarrow #2}}

\newcommand{\hGamma}{\dot{\Gamma}}
\newcommand{\domof}[1]{\text{dom}(#1)}
\newcommand{\hsyn}[3]{#1 \vdash #2 \Rightarrow #3}
\newcommand{\hana}[3]{#1 \vdash #2 \Leftarrow #3}
\newcommand{\hexptyp}[4]{#1 \mathbin{;} #2 \vdash #3 : #4}
\newcommand{\hpattyp}[4]{#1 : #2 \dashV #3 \mathbin{;} #4}
\newcommand{\hsubstyp}[2]{#1 : #2}
\newcommand{\hpatmatch}[3]{#1 \vartriangleright #2 \dashV #3}
\newcommand{\hnotmatch}[2]{#1 \mathbin{\bot} #2}
\newcommand{\hmaymatch}[2]{#1 \mathbin{?} #2}
\newcommand{\htrans}[2]{#1 \mapsto #2}

\newcommand{\isVal}[1]{#1 ~\mathtt{val}}
\newcommand{\isErr}[1]{#1 ~\mathtt{err}}
\newcommand{\isIndet}[1]{#1 ~\mathtt{indet}}
\newcommand{\isFinal}[1]{#1 ~\mathtt{final}}

% ZTyp and ZExp
\newcommand{\zlsel}[1]{{\bowtie}{#1}}
\newcommand{\zrsel}[1]{{#1}{\bowtie}}
\newcommand{\zwsel}[1]{
  \setlength{\fboxsep}{0pt}
  \colorbox{green!10!white!100}{
    \ensuremath{{{\textcolor{Green}{{\hspace{-2px}\triangleright}}}}{#1}{\textcolor{Green}{\triangleleft{\vphantom{\tehole}}}}}}
}

\newcommand{\removeSel}[1]{#1^{\diamond}}

% ZTyp
\newcommand{\ztau}{\hat{\tau}}

% ZExp
\newcommand{\zexp}{\hat{e}}

% Direction
\newcommand{\dParent}{\mathtt{parent}}
\newcommand{\dChildn}[1]{\mathtt{child}~\mathtt{{#1}}}
\newcommand{\dChildnm}[1]{\mathtt{child}~{#1}}

% Action
\newcommand{\aMove}[1]{\mathtt{move}~#1}
	\newcommand{\zrightmost}[1]{\mathsf{rightmost}(#1)}
	\newcommand{\zleftmost}[1]{\mathsf{leftmost}(#1)}
\newcommand{\aSelect}[1]{\mathtt{sel}~#1}
\newcommand{\aDel}{\mathtt{del}}
\newcommand{\aReplace}[1]{\mathtt{replace}~#1}
\newcommand{\aConstruct}[1]{\mathtt{construct}~#1}
\newcommand{\aConstructx}[1]{#1}
\newcommand{\aFinish}{\mathtt{finish}}

\newcommand{\performAna}[5]{#1 \vdash #2 \xlongrightarrow{#4} #5 \Leftarrow #3}
\newcommand{\performAnaI}[5]{#1 \vdash #2 \xlongrightarrow{#4}\hspace{-3px}{}^{*}~ #5 \Leftarrow #3}
\newcommand{\performSyn}[6]{#1 \vdash #2 \Rightarrow #3 \xlongrightarrow{#4} #5 \Rightarrow #6}
\newcommand{\performSynI}[6]{#1 \vdash #2 \Rightarrow #3 \xlongrightarrow{#4}\hspace{-3px}{}^{*}~ #5 \Rightarrow #6}
\newcommand{\performTyp}[3]{#1 \xlongrightarrow{#2} #3}
\newcommand{\performTypI}[3]{#1 \xlongrightarrow{#2}\hspace{-3px}{}^{*}~#3}

\newcommand{\performMove}[3]{#1 \xlongrightarrow{#2} #3}
\newcommand{\performDel}[2]{#1 \xlongrightarrow{\aDel} #2}

% Form
\newcommand{\farr}{\mathtt{arrow}}
\newcommand{\fnum}{\mathtt{num}}
\newcommand{\fsum}{\mathtt{sum}}

\newcommand{\fasc}{\mathtt{asc}}
\newcommand{\fvar}[1]{\mathtt{var}~#1}
\newcommand{\flam}[1]{\mathtt{lam}~#1}
\newcommand{\fap}{\mathtt{ap}}
% \newcommand{\farg}{\mathtt{arg}}
\newcommand{\fnumlit}[1]{\mathtt{lit}~#1}
\newcommand{\fplus}{\mathtt{plus}}
\newcommand{\fhole}{\mathtt{hole}}
\newcommand{\fnehole}{\mathtt{nehole}}

\newcommand{\finj}[1]{\mathtt{inj}~#1}
\newcommand{\fcase}[2]{\mathtt{case}~#1~#2}

% Talk about formal rules in example
\newcommand{\refrule}[1]{\textrm{Rule~(#1)}}

\newcommand{\herase}[1]{\left|#1\right|_\textsf{erase}}

\newcommand{\arrmatch}[2]{#1 \blacktriangleright_{\rightarrow} #2}


\newcommand{\TABperformAna}[5]{#1 \vdash & #2                & \xlongrightarrow{#4} & #5 & \Leftarrow #3}
\newcommand{\TABperformSyn}[6]{#1 \vdash & #2 \Rightarrow #3 & \xlongrightarrow{#4} & #5 \Rightarrow #6}
\newcommand{\TABperformTyp}[3]{& #1 & \xlongrightarrow{#2} & #3}

\newcommand{\TABperformMove}[3]{#1 & \xlongrightarrow{#2} & #3}
\newcommand{\TABperformDel}[2]{#1 \xlongrightarrow{\aDel} #2}

\newcommand{\sumhasmatched}[2]{#1 \mathrel{\textcolor{black}{\blacktriangleright_{+}}} #2}

\newcommand{\subminsyn}[1]{\mathsf{submin}_{\Rightarrow}(#1)}
\newcommand{\subminana}[1]{\mathsf{submin}_{\Leftarrow}(#1)}


\newcommand{\inparens}[1]{{\color{gray}(}#1{\color{gray})}}

%% rule names for appendix
\newcommand{\rname}[1]{\textsc{#1}}
\newcommand{\gap}{\vspace{7pt}}


\setlength{\abovecaptionskip}{4pt plus 3pt minus 2pt} % Chosen fairly arbitrarily
\setlength{\belowcaptionskip}{-4pt plus 3pt minus 2pt} % Chosen fairly arbitrarily


\begin{document}

%% Title information
\title{patterns paper}         %% [Short Title] is optional;
\ifarxiv
\subtitle{Extended Version}
\subtitlenote{The original version of this article was published in the POPL 2019 edition of PACMPL \cite{HazelnutLive}. This extended version includes an additional appendix.}
\fi
                                        %% when present, will be used in

                                        %% header instead of Full Title.
% \titlenote{with title note}             %% \titlenote is optional;
                                        %% can be repeated if necessary;
                                        %% contents suppressed with 'anonymous'
% \subtitle{Subtitle}                     %% \subtitle is optional
% \subtitlenote{with subtitle note}       %% \subtitlenote is optional;
                                        %% can be repeated if necessary;
                                        %% contents suppressed with 'anonymous'


%% Author information
%% Contents and number of authors suppressed with 'anonymous'.
%% Each author should be introduced by \author, followed by
%% \authornote (optional), \orcid (optional), \affiliation, and
%% \email.
%% An author may have multiple affiliations and/or emails; repeat the
%% appropriate command.
%% Many elements are not rendered, but should be provided for metadata
%% extraction tools.

%% Author with single affiliation.
% \author{Cyrus Omar}
% \authornote{with author1 note}          %% \authornote is optional;
                                        %% can be repeated if necessary
% \orcid{nnnn-nnnn-nnnn-nnnn}             %% \orcid is optional
% \affiliation{
  % \position{Position1}
  % \department{Department1}              %% \department is recommended
  % \institution{University of Chicago, USA}            %% \institution is required
  % \streetaddress{Street1 Address1}
  % \city{City1}
  % \state{State1}
  % \postcode{Post-Code1}
  % \country{Country1}
% }
% \email{comar@cs.uchicago.edu}          %% \email is recommended


% %% Author with two affiliations and emails.
% \author{First2 Last2}
% \authornote{with author2 note}          %% \authornote is optional;
%                                         %% can be repeated if necessary
% \orcid{nnnn-nnnn-nnnn-nnnn}             %% \orcid is optional
% \affiliation{
%   \position{Position2a}
%   \department{Department2a}             %% \department is recommended
%   \institution{Institution2a}           %% \institution is required
%   \streetaddress{Street2a Address2a}
%   \city{City2a}
%   \state{State2a}
%   \postcode{Post-Code2a}
%   \country{Country2a}
% }
% \email{first2.last2@inst2a.com}         %% \email is recommended
% \affiliation{
%   \position{Position2b}
%   \department{Department2b}             %% \department is recommended
%   \institution{Institution2b}           %% \institution is required
%   \streetaddress{Street3b Address2b}
%   \city{City2b}
%   \state{State2b}
%   \postcode{Post-Code2b}
%   \country{Country2b}
% }
% \email{first2.last2@inst2b.org}         %% \email is recommended


%% Paper note
%% The \thanks command may be used to create a "paper note" ---
%% similar to a title note or an author note, but not explicitly
%% associated with a particular element.  It will appear immediately
%% above the permission/copyright statement.
% \thanks{with paper note}                %% \thanks is optional
                                        %% can be repeated if necesary
                                        %% contents suppressed with 'anonymous'


%% Abstract
%% Note: \begin{abstract}...\end{abstract} environment must come
%% before \maketitle command
% % !TEX root = hazel-LIVE2018.tex

\begin{abstract}
\emph{Type inference} allows programmers to omit type annotations while still permitting type checking. The problem is that \emph{type inference} only works on complete programs, e.g. syntactically well-formed expressions. This paper introduces a \emph{type hole inference} system that applies type inference on incomplete programs. We start from modeling incomplete programs by \emph{holes} developed in \HazelnutLive, where holes stand for missing types and expressions. Rather than obtaining limited static type information for type holes from bidirectional typing propagation, we extended static semantics with \emph{type constraints}, and borrow machinery from unification-based type inference to infer types for type holes. Our approach shares benefits of \emph{bidirectional typing}, which produces good-quality error messages, and unification, which infers types for unknown parts of incomplete programs to collect as much static type information as possible. 


\end{abstract}
    


%% 2012 ACM Computing Classification System (CSS) concepts
%% Generate at 'http://dl.acm.org/ccs/ccs.cfm'.
% \begin{CCSXML}
% <ccs2012>
% <concept>
% <concept_id>10011007.10011006.10011008</concept_id>
% <concept_desc>Software and its engineering~General programming languages</concept_desc>
% <concept_significance>500</concept_significance>
% </concept>
% <concept>
% <concept_id>10003456.10003457.10003521.10003525</concept_id>
% <concept_desc>Social and professional topics~History of programming languages</concept_desc>
% <concept_significance>300</concept_significance>
% </concept>
% </ccs2012>
% \end{CCSXML}
\begin{CCSXML}
<ccs2012>
<concept>
<concept_id>10011007.10011006.10011008.10011009.10011012</concept_id>
<concept_desc>Software and its engineering~Functional languages</concept_desc>
<concept_significance>500</concept_significance>
</concept>
% <concept>
% <concept_id>10003752.10010124.10010131.10010134</concept_id>
% <concept_desc>Theory of computation~Operational semantics</concept_desc>
% <concept_significance>500</concept_significance>
% </concept>
</ccs2012>
\end{CCSXML}

\ccsdesc[500]{Software and its engineering~Functional languages}
% \ccsdesc[500]{Theory of computation~Operational semantics}
%% End of generated code


%% Keywords
%% comma separated list
% \keywords{live programming, gradual typing, contextual modal type theory, typed holes, structured editing}  %% \keywords is optional


%% \maketitle
%% Note: \maketitle command must come after title commands, author
%% commands, abstract environment, Computing Classification System
%% environment and commands, and keywords command.
% \maketitle
% \thispagestyle{empty} % suppresses the footer

% \section{Introduction}
\label{sec:intro}

% \input{examples}
% \input{calculus}
% \input{implementation}
% \input{related-work}
% \input{discussion}

% !TEX root = ./type-inference-paper.tex
\begin{figure}[t]
$\arraycolsep=4pt\begin{array}{lll}
\tau & ::= &
  \tnum ~\vert~
  \tarr{\tau}{\tau} ~\vert~
  ? ~\vert~
  \ehole^A
  \\
e & ::= &
  x ~\vert~
  \lamfunc{x}{e} ~\vert~
  e(e) ~\vert~
  \underlinenum{n} ~\vert~
  (e+e) ~\vert~
  e: \tau ~\vert~
  \ehole^u  ~\vert~
  \notehole{e}^u \\
\end{array}$
\end{figure}

\begin{figure}[t]
  \fbox{$\consexptyp{\Gamma}{e}{\tau}{C}$}~~\text{$e$ synthesizes $\tau$}
\begin{subequations}
\begin{equation}
\inferrule[SVar]{ }{
  \consexptyp{\Gamma, x : \tau}{x}{\tau}{\econs}
}
\end{equation}
\begin{equation}
\inferrule[SNum]{ }{
  \consexptyp{\Gamma}{\hnum{n}}{\tnum}{\econs}
}
\end{equation}

\begin{equation}
\inferrule[SAp]{
  \consexptyp{\Gamma}{e_1}{\tau_1}{C_1} \\
  \tau_1 \typearrow \tarr{\tau_{in}}{\tau_{out}} \addcons{C_2} \\
  \ana{\Gamma}{e_2}{\tau_{in}}{C_3}
}{
  \consexptyp{\Gamma}{\hap{e_1}{e_2}}{\tau_{out}} { C_1 \cup C_2 \cup C_3}
}
\end{equation}
\begin{equation}
\inferrule[SPlus]{
  \ana{\Gamma}{e}{\tnum}{C_1} \\
  \ana{\Gamma}{e}{\tnum}{C_2}
}{
  \consexptyp{\Gamma}{(e_1 + e_2)}{\tnum}{C_1 \cup C_2}
}
\end{equation}

\begin{equation}
\inferrule[*]{
  \ana{\Gamma}{e}{\tau}{C}
}{
  \consexptyp{\Gamma}{(e : \tau)}{\tau}{C}
}
\end{equation}

\begin{equation}
\inferrule[SEHole]{(A~fresh) }{
  \consexptyp{\Gamma}{\llparenthesiscolor \rrparenthesiscolor^u}{\llparenthesiscolor \rrparenthesiscolor^A}{\econs}
}
\end{equation}
\begin{equation}
\inferrule[SHole]{
 (A~fresh) \\
 \consexptyp{\Gamma}{e}{\tau}{C}
}{
  \consexptyp{\Gamma}{\llparenthesiscolor e \rrparenthesiscolor^u}{\llparenthesiscolor \rrparenthesiscolor^A}{C}
}
\end{equation}
\end{subequations}
\end{figure}

\begin{figure}[t]
\fbox{$\ana{\Gamma}{e}{\tau} {C}$}~~\text{$e$ analyzes against $\tau$}
\begin{subequations}
\begin{equation}
\inferrule[Analyzation]{ }{
  \ana{\Gamma}{e:\tau_1}{\tau_2}{\{ \tau_2 \sqsubseteq \tau_1 \}}
}
\end{equation}

\begin{equation}
\inferrule[ALam]{
 \tau \typearrow \tarr{\tau_{in}}{\tau_{out}} \addcons{C_1} \\
  \ana{\Gamma, x : \tau_{in}}{e}{\tau_{in}}{C_2}
}{
  \ana{\Gamma}{\lamfunc{x}{e}}{\tau}{C_1 \cup C_2}
}
\end{equation}

\begin{equation}
\inferrule[ASubsumption]{
  \consexptyp{\Gamma}{e}{\tau'}{C_1} \\
  \tau \sqsubseteq \tau' 
}{
  \ana{\Gamma}{e}{\tau}{C_1 \cup \{\tau \sqsubseteq \tau'  \}}
}
\end{equation}


\end{subequations}
\end{figure}


\begin{figure}[t]
\fbox{$\tau \typearrow \tarr{\tau_{in}}{\tau_{out}} \addcons{C}$}~~\text{$\tau$ is equivalent to $\tarr{\tau_{in}}{\tau_{out}}$}
\begin{subequations}

\begin{equation}
\inferrule[EType]{ }{
  \tarr{\tau_{in}}{\tau_{out}} \typearrow \tarr{\tau_{in}}{\tau_{out}} \addcons{\econs}
}
\end{equation}

\begin{equation}
\inferrule[EHole]{
 (B~fresh) \\
 (C~fresh)
}{
  \tehole^A \typearrow \tarr{\tehole^B}{\tehole^C} \addcons{\{ A \simeq B \rightarrow C \} }
}
\end{equation}

\end{subequations}
\end{figure}

\begin{figure}[t]
\fbox{$\tau \sqsubseteq \tau' $}~~\text{$\tau$ is consistent-less than $\tau'$}
\begin{subequations}

\begin{equation}
\inferrule[CLNum]{ }{
  \conless{num}{num}
}
\end{equation}
\begin{equation}
\inferrule[CLG]{ }{
  \conless{\ehole^A}{\ehole^A}
}
\end{equation}
\begin{equation}
\inferrule[CLDL]{ }{
  \conless{?}{\tau}
}
\end{equation}

\begin{equation}
\inferrule[CLFun]{
\conless{\tau_1}{\tau_3} \\
\conless{\tau_2}{\tau_4}
 }{
  \conless{\tarr{\tau_1}{\tau_2}}{\tarr{\tau_3}{\tau_4}}
}
\end{equation}


\end{subequations}
\end{figure}


\begin{figure}[t]
\fbox{$\tau \simeq \tau $}~~\text{$\tau$ is consistent equivalent to $\tau$}
\begin{subequations}

\begin{equation}
\inferrule[CENum]{
}{
  num \simeq num
}
\end{equation}

\begin{equation}
\inferrule[CEG]{
}{
  \ehole^A \simeq \ehole^A
}
\end{equation}

\begin{equation}
\inferrule[CEDL]{
}{
  ? \simeq \tau
}
\end{equation}

\begin{equation}
\inferrule[CEDR]{
}{
  \tau \simeq ?
}
\end{equation}

\begin{equation}
\inferrule[CEFun]{
\tau_1 \simeq \tau_3 \\
\tau_2 \simeq \tau_4
}{
  \tarr{\tau_1}{\tau_2} \simeq \tarr{\tau_3}{\tau_4}
}
\end{equation}

\begin{equation}
\inferrule[CEVL]{
\conless{\tau_1}{\tau_2}
}{
  \tau_1 \simeq \tau_2
}
\end{equation}

\begin{equation}
\inferrule[CEVR]{
\conless{\tau_1}{\tau_2}
}{
  \tau_2 \simeq \tau_1
}
\end{equation}

\end{subequations}
\end{figure}


\begin{thm}[Preservation]
  \label{thrm:preservation}
  If $\tau \simeq \tau' \addcons{C}$ then $\tau \simeq \tau'$
\end{thm}

\begin{thm}[Constraint Existence]
  \label{Constraint Existence}
  If $\Gamma \vdash e \Rightarrow \tau$ then $\consexptyp{\Gamma}{e}{\tau}{C}$ for some C
\end{thm}
%\clearpage
\bibliography{all.short}

\ifarxiv
\clearpage
\appendix
%\input{appendix-defns}
%\input{extensions}
\else

\fi
% \input{misc}
% \input{implementation-appendix}

\end{document}
